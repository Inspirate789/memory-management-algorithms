\part*{ВВЕДЕНИЕ}
\addcontentsline{toc}{part}{\textbf{ВВЕДЕНИЕ}}

Память является одним из основных ресурсов любой вычислительной системы, требующих тщательного управления. Под памятью (memory) в работе будет подразумеваться оперативная память компьютера. Особая роль памяти объясняется тем, что процессор может выполнять инструкции программы только в том случае, если они находятся в памяти. Память распределяется между операционной системой компьютера и прикладными программами. \cite{tannenbaum}

Для выполнения вычислений в языках программирования используются объекты, которые могут быть представлены как простым типом данных (целые числа, символы, логические значения и т.д.) так и агрегированным (массивы, списки, деревья и т.д.). Значения объектов программ хранятся в памяти для быстрого доступа. Во многих языках программирования переменная в программном коде --- это адрес объекта в памяти. \cite{c} \cite{cpp} \cite{golang} Когда переменная используется в программе, процесс считывает значение из памяти и обрабатывает его.

Большинство современных языков программирования активно использует динамическое распределение памяти, при котором выделение объектов осуществляется во время выполнения программы. Динамическое управление памятью вводит два основных примитива --- функции выделения и освобождения памяти, за которые отвечает \textbf{аллокатор}. 

Существует два способа управления динамической памятью --- ручное и автоматическое. При ручном управлении памятью программист должен следить за освобождением выделенной памяти, что приводит к возможности возникновения ошибок. Более того, в некоторых ситуациях (например, при программировании на функциональных языках или в многопоточной среде) время жизни объекта не всегда очевидно для разработчика. \cite{elixir} Автоматическое управление памятью избавляет программиста от необходимости вручную освобождать выделенную память, устраняя тем самым целый класс возможных ошибок и увеличивая безопасность разрабатываемых программ. Сборка мусора (garbage collection) за последние два десятилетия стала стандартом в области автоматического управления памятью, хотя её использование накладывает дополнительные расходы по памяти и времени исполнения. На сегодняшний день среды времени выполнения (language runtime) многих популярных языков программирования, таких как Java, C\#, Python и другие, активно используют сборку мусора. 

Целью данной работы является изучение алгоритмов распределения памяти в языках программирования с автоматической сборкой мусора. Для достижения поставленной цели необходимо решить следующие задачи.

\begin{enumerate}[label*=\arabic*.]
	\item Проанализировать предметную область работы с памятью в языках программирования с автоматической сборкой мусора.
	\item Рассмотреть существующие принципы организации работы с памятью в языках программирования с автоматической сборкой мусора на примере Python, Java, JavaScript, C\# и Golang.
	\item Описать алгоритмы сборки мусора в рассматриваемых языках.
	\item Сформулировать критерии сравнения и оценки описанных алгоритмов.
	\item Сравнить существующие решения по сформулированным критериям. 
\end{enumerate}